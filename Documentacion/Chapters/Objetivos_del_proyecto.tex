\chapter{Objetivos del proyecto}

%\newpage
%Objetivos, alcance, planificación temporal, herramientas, 
%gestión de riesgos, evaluación económica

\section{Objetivos}
Los objetivos del proyecto consisten en crear un software que extienda la funcionalidad actual del sistema INTRASIM desarrollado por la universidad de Deusto y del grupo de investigación Galan perteneciente a la UPV/EHU.

Es un software que permite gráficamente visualizar las observaciones \footnote{Aquí va una explicación de lo que es una observación} del sistema. El objetivo principal es brindar al experto una herramienta para poder señalar gráficamente en que momento de esa observación sucede el hecho concreto que estaban buscando.

La herramienta permite visualizar los datos en bruto tanto en modo v\'{i}deo (si esta disponible), como en modo gr\'{a}fico, con polilineas, histogramas etc. Tales gráficos permiten la selección de rangos.

\section{Alcance}
Aquí va el alcance del proyecto.

\section{Planificaci\'{o}n temporal}
En vez de seguir una planificación clásica, como puede ser COCOMO \footnote{\url{https://en.wikipedia.org/wiki/COCOMO}} o Waterfall \footnote{\url{https://en.wikipedia.org/wiki/Waterfall_model}} he decidido utilizar las metodologías agiles de desarrollo, concretamente Scrum y Kanban

\subsection{Scrum}
Scrum es una serie de herramientas (framework) para la gestión y desarrollo de software basado en un proceso iterativo e incremental. \emph{cita a la wikipedia}

\begin{figure}[h]
    \includegraphics[width=0.7\linewidth]{./Figures/Scrumm}
    \caption[Proceso iterativo Scrum]{En este gráfico podemos ver el método de trabajo según Scrum, en el cual se observa lo importante que son los entregables}
    \label{fig:Scrum}
\end{figure}

%\newpage
Scrum, de manera resumida consiste en lo siguiente:
\begin{enumerate}
    \item Hablar con el cliente para ver si hay nuevos requisitos.
    \item Crear backlog, es decir, las tareas de ese sprint.
    \item Programar los requisitos especificados en el sprint.
    \item Reunión de retrospectiva para ver que ha ido mal y mejorarlo.
    \item Enviar entregable al cliente.
    \item Repetir.
\end{enumerate}

Es fundamental entre sprint y sprint que el valor del producto haya aumentado. Es decir, lo importante es el producto, no como vamos nosotros en el propio proyecto. Podríamos haber avanzado mucho en el diseño del software, pero si eso es algo que el cliente no puede ver estaremos fallando en nuestra agilidad.

Por otro lado, si cometemos algún error será muy sencillo corregir el error ya que si los sprints son de dos semanas por ejemplo, no habremos perdido apenas tiempo. Y además para el cliente los entregables serán predecibles en el tiempo.

Scrum es también muy importante, y se posiciona como una alternativa muy potente frente a COCOMO o waterfall, es porque en la creación de software hay un componente de incertidumbre muy grande. No sabemos como, ni cuando pueden cambiar los requisitos de un software. Por ello toman importancia los sprints de nuevo. 

En los modelos antiguos, si ya habíamos terminado la fase de análisis y diseño, y se requería una nueva funcionalidad es necesario paralizar el desarrollo del software y volver a analizar y diseñar. En cambio, con Scrum es posible integrar ese cambio en el siguiente sprint.

\subsection{Kanban}
Kanban es un método de organización del conocimiento del trabajo que se está realizando con un gran énfasis en el la entrega justo a tiempo \footnote{JIT Delivery: Just-In-Time Delivery} sin sobrecargar al equipo \emph{Cita a la wikipedia}

Kanban se centra mucho en que lo importante no es empezar muchas cosas, si no que acabemos aquello que empezamos. Sirve sobre todo para ver de una manera visual:
\begin{itemize}
    \item Qué tenemos pendiente
    \item Qué estamos haciendo ahora mismo
    \item Qué hemos terminado.
\end{itemize}

El método Kanban es algo muy extenso, pero yo simplemente he aplicado el tablero Kanban mezclado con Scrum. Cada dos semanas, me creo una lista de tareas que debo completar (Backlog) y lo coloco en la columna de "Tareas pendientes", ordenadas por prioridad. Después, coloco en la lista de "En progreso" como máximo dos tareas. Y no empiezo ninguna otra hasta que esas dos hayan sido pasadas a la columna de "Tareas finalizadas".

\section{Herramientas}
El proyecto, al no ser enteramente mío, no he tenido una libertad total para la elección de herramientas. 

Las herramientas que he utilizado para llevar a cabo este proyecto han sido:
\begin{itemize}
    \item 
    \textbf{Visual Studio}
    
    Para el desarrollo propiamente dicho del proyecto he usado Visual Studio 2013 Ultimate, proporcionado de manera gratuita gracias al acuerdo que mantiene la UPV/EHU con Microsoft.
    
    \item 
    \textbf{TeXstudio y \LaTeX}
    
    Dos herramientas gratuitas y de código abierto para generar documentos.
    
    \item
    \textbf{Git y GitHub}
    
    Fundamental en los proyectos que se basen en metodologías ágiles. Git es un software de gestión de control de versiones distribuido, esto es, que cada desarrollador dispone de una copia completa del repositorio, desarrollado por Linus Torvalds y Junio Hamano en 2005 y similar a SVN.
\end{itemize}

\section{Gesti\'{o}n de riesgos}
Dadas las fechas de entrega impuestas por mi mismo para la finalizaci\'{o}n del Trabajo de Fin de Grado, es un hecho que
aparecerán riesgos que pueden poner en peligro el proyecto. Es por ello que se necesita tener en cuenta las probabilidades de los sucesos que 
pueden ocurrir y que puedan retrasar el trabajo de forma notable. Una vez
listados, se puede proceder a crear un plan de prevención para evitarlos. Y en caso de que la prevención no
funcionara, un plan de contingencia que pueda amortiguar las consecuencias de ese riesgo. A continuación
se listan de forma detallada los riesgos que pueden aparecer durante el transcurso del proyecto.

\begin{tabular}{l*{1}{c}r}
    Concepto                   & Gastos en € / mes & \\
    \hline
    Luz                        & 60 & \\
    Agua                       & 50 & \\
    Internet + Teléfono        & 60 & \\
    Gas                        & 70 & \\
    Seguro                     & 45 & \\
    IBI                        & 40 & \\
    \hline
    Total Mes                  & 325 & \\
    Total A\~{n}o              & 3900 &\\
\end{tabular}

\section{Viabilidad}
El presente proyecto se ha realizado en colaboraci\'{o}n del grupo GALAN de la UPV/EHU. Concretamente ha sido un proyecto propuesto por ellos
y se entiende que es viable.

\section{Planificaci\'{o}n econ\'{o}mica}
Como el proyecto es en colaboraci\'{o}n de un grupo de investigaci\'{o}n de la UPV/EHU no es necesario estimar los costes ya que
el proyecto se esta realizando bajo el amparo de tal grupo, y se est\'{a} realizando por el bien de la ciencia y del progreso.

%Nuestra empresa la forman:
%\begin{itemize}
%    \item
%    \textbf{Ángel Agudo}: CEO \footnote{Chief Executive Officer: Director ejecutivo} de la empresa.
%    \item
%    \textbf{Asun Santos}: Directora de calidad y administración.
%    \item
%    \textbf{Elena Agudo}: Directora de arte.
%    \item
%    \textbf{Rubén Agudo}: CTO \footnote{Chief Technical Officer: Director de tecnología} y responsable de I+D. 
%\end{itemize}
%Cómo es una epoca complicada por la crisis, tanto Ángel como Asun se han visto en la necesidad de pluriemplearse como administrativos de Osakidetza. 

%Su empleo en Osakidetza reportan a la empresa familiar una cantidad neta de 5000€ mensuales. Nuestro único proyecto de software es este proyecto de fin de %carrera por lo que todos los ingresos provienen de sus otros empleos.

%Como somos una empresa muy concienciada con la investigación y el desarrollo hemos decidido que el proyecto sea gratis y de código abierto, licenciado bajo GPLv3.

%Por lo tanto el coste de venta de nuestro coste sera 0€

%\subsection{Amortizaci\'{o}n de material}
%\subsubsection{Oficina}
%La oficina que utilizamos es el domicilio familiar y no hay que pagar alquiler ya que es de nuestra propiedad. Por lo que el gasto de alquiler/hipoteca es 0€
%\subsubsection{Material inform\'{a}tico}
%El material informático que se ha usado ha sido un portátil Acer TravelMate 5742G comprado en 
%Diciembre de 2010 y usado intensivamente durante estos 4 años. 
%El portátil estaba pensado para ser amortizado en 4 años. \emph{As\'{i} que por aqui habra que hacer algun calculo tonto}

%\subsection{Otros gastos}

%\begin{tabular}{l*{1}{c}r}
%    Concepto                   & Gastos en € / mes & \\
%    \hline
%    Luz                        & 60 & \\
%    Agua                       & 50 & \\
%    Internet + Teléfono        & 60 & \\
%    Gas                        & 70 & \\
%    Seguro                     & 45 & \\
%    IBI                        & 40 & \\
%    \hline
%    Total Mes                  & 325 & \\
%    Total A\~{n}o              & 3900 &\\
%\end{tabular}


%\subsection{Salarios}
%De los 5000€ que se ingresan mensualmente se reparten de la siguiente manera:
%\newline
%\newline
%\begin{tabular}{l*{1}{c}r}
%    Miembro de la empresa     & Salario en € & \\
%    \hline
%    \'{A}ngel Agudo               & 1000 & \\
%    Asun Santos               & 1000 & \\
%    Elena Agudo               & 80 & \\
%    Rubén Agudo               & 80 & \\
%    \hline
%    Remanente Empresa         & 2840 & \\
%\end{tabular}

